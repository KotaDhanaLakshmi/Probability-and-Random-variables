\let\negmedspace\undefined 
 \let\negthickspace\undefined 
\documentclass[journal,12pt,onecolumn]{IEEEtran} 
 %\documentclass[conference]{IEEEtran} 
 %\IEEEoverridecommandlockouts 
 % The preceding line is only needed to identify funding in the first footnote. If that is unneeded, please comment it out. 
 \usepackage{cite} 
 \usepackage{amsmath,amssymb,amsfonts,amsthm} 
 \usepackage{algorithmic} 
 \usepackage{graphicx} 
 \usepackage{textcomp} 
 \usepackage{xcolor} 
 \usepackage{txfonts} 
 \usepackage{listings} 
 \usepackage{enumitem} 
 \usepackage{mathtools} 
 \usepackage{gensymb} 
 \usepackage[breaklinks=true]{hyperref} 
 \hypersetup{
    colorlinks=true,
    linkcolor=blue,
    urlcolor=blue,
    filecolor=magenta,
    citecolor=blue,
    urlcolor=blue,
}

 \usepackage{tkz-euclide} % loads  TikZ and tkz-base 
 \usepackage{listings} 
 \usepackage{caption}
 % 
 %\usepackage{setspace} 
 %\usepackage{gensymb} 
 %\doublespacing 
 %\singlespacing 
  
 %\usepackage{graphicx} 
 %\usepackage{amssymb} 
 %\usepackage{relsize} 
  %\usepackage[cmex10]{amsmath} 
 %\usepackage{amsthm} 
 %\interdisplaylinepenalty=2500 
 %\savesymbol{iint} 
 %\usepackage{txfonts} 
 %\restoresymbol{TXF}{iint} 
 %\usepackage{wasysym} 
 %\usepackage{amsthm} 
 %\usepackage{iithtlc} 
 %\usepackage{mathrsfs} 
 %\usepackage{txfonts} 
 %\usepackage{stfloats} 
 %\usepackage{bm} 
 %\usepackage{cite} 
 %\usepackage{cases} 
 %\usepackage{subfig} 
 %\usepackage{xtab} 
 %\usepackage{longtable} 
 %\usepackage{multirow} 
 %\usepackage{algorithm} 
 %\usepackage{algpseudocode} 
 %\usepackage{enumitem} 
 %\usepackage{mathtools} 
 %\usepackage{tikz} 
 %\usepackage{circuitikz} 
 %\usepackage{verbatim} 
 %\usepackage{tfrupee} 
 %\usepackage{stmaryrd} 
 %\usetkzobj{all} 
 %    \usepackage{color}                                            %% 
 %    \usepackage{array}                                            %% 
 %    \usepackage{longtable}                                        %% 
 %    \usepackage{calc}                                             %% 
 %    \usepackage{multirow}                                         %% 
 %    \usepackage{hhline}                                           %% 
 %    \usepackage{ifthen}                                           %% 
   %optionally (for landscape tables embedded in another document): %% 
 %    \usepackage{lscape}      
 %\usepackage{multicol} 
 %\usepackage{chngcntr} 
 %\usepackage{enumerate} 
  
 %\usepackage{wasysym} 
 %\newcounter{MYtempeqncnt} 
 \DeclareMathOperator*{\Res}{Res} 
 %\renewcommand{\baselinestretch}{2} 
 \renewcommand\thesection{\arabic{section}} 
 \renewcommand\thesubsection{\thesection.\arabic{subsection}} 
 \renewcommand\thesubsubsection{\thesubsection.\arabic{subsubsection}} 
  
 \renewcommand\thesectiondis{\arabic{section}} 
 \renewcommand\thesubsectiondis{\thesectiondis.\arabic{subsection}} 
 \renewcommand\thesubsubsectiondis{\thesubsectiondis.\arabic{subsubsection}} 
  
 % correct bad hyphenation here 
 \hyphenation{op-tical net-works semi-conduc-tor} 
 \def\inputGnumericTable{}                                 %% 
  
 \lstset{ 
 %language=C, 
 frame=single,  
 breaklines=true, 
 columns=fullflexible 
 } 
 %\lstset{ 
 %language=tex, 
 %frame=single,  
 %breaklines=true 
 %} 
  
 \begin{document} 
 % 
  
  
 \newtheorem{theorem}{Theorem}[section] 
 \newtheorem{problem}{Problem} 
 \newtheorem{proposition}{Proposition}[section] 
 \newtheorem{lemma}{Lemma}[section] 
 \newtheorem{corollary}[theorem]{Corollary} 
 \newtheorem{example}{Example}[section] 
 \newtheorem{definition}[problem]{Definition} 
 %\newtheorem{thm}{Theorem}[section]  
 %\newtheorem{defn}[thm]{Definition} 
 %\newtheorem{algorithm}{Algorithm}[section] 
 %\newtheorem{cor}{Corollary} 
 \newcommand{\BEQA}{\begin{eqnarray}} 
 \newcommand{\EEQA}{\end{eqnarray}} 
 \newcommand{\define}{\stackrel{\triangle}{=}} 
  
 \bibliographystyle{IEEEtran} 
 %\bibliographystyle{ieeetr} 
  
  
 \providecommand{\mbf}{\mathbf} 
 \providecommand{\pr}[1]{\ensuremath{\Pr\left(#1\right)}} 
 \providecommand{\qfunc}[1]{\ensuremath{Q\left(#1\right)}} 
 \providecommand{\sbrak}[1]{\ensuremath{{}\left[#1\right]}} 
 \providecommand{\lsbrak}[1]{\ensuremath{{}\left[#1\right]}} 
 \providecommand{\rsbrak}[1]{\ensuremath{{}\left[#1\right]}} 
 \providecommand{\brak}[1]{\ensuremath{\left(#1\right)}} 
 \providecommand{\lbrak}[1]{\ensuremath{\left(#1\right.}} 
 \providecommand{\rbrak}[1]{\ensuremath{\left.#1\right)}} 
 \providecommand{\cbrak}[1]{\ensuremath{\left\{#1\right\}}} 
 \providecommand{\lcbrak}[1]{\ensuremath{\left\{#1\right.}} 
 \providecommand{\rcbrak}[1]{\ensuremath{\left.#1\right\}}} 
 \theoremstyle{remark} 
 \newtheorem{rem}{Remark} 
 \newcommand{\sgn}{\mathop{\mathrm{sgn}}} 
 \providecommand{\abs}[1]{\left\vert#1\right\vert} 
 \providecommand{\res}[1]{\Res\displaylimits_{#1}}  
 \providecommand{\norm}[1]{\left\lVert#1\right\rVert} 
 %\providecommand{\norm}[1]{\lVert#1\rVert} 
 \providecommand{\mtx}[1]{\mathbf{#1}} 
 \providecommand{\mean}[1]{E\left[ #1 \right]} 
 \providecommand{\fourier}{\overset{\mathcal{F}}{ \rightleftharpoons}} 
 %\providecommand{\hilbert}{\overset{\mathcal{H}}{ \rightleftharpoons}} 
 \providecommand{\system}{\overset{\mathcal{H}}{ \longleftrightarrow}} 
         %\newcommand{\solution}[2]{\textbf{Solution:}{#1}} 
 \newcommand{\solution}{\noindent \textbf{Solution: }} 
 \newcommand{\cosec}{\,\text{cosec}\,} 
 \providecommand{\dec}[2]{\ensuremath{\overset{#1}{\underset{#2}{\gtrless}}}} 
 \newcommand{\myvec}[1]{\ensuremath{\begin{pmatrix}#1\end{pmatrix}}} 
 \newcommand{\mydet}[1]{\ensuremath{\begin{vmatrix}#1\end{vmatrix}}} 
 %\numberwithin{equation}{section} 
 %\numberwithin{equation}{subsection} 
 %\numberwithin{problem}{section} 
 %\numberwithin{definition}{section} 
 %\makeatletter 
 %\@addtoreset{figure}{problem} 
 %\makeatother 
  
 %\let\StandardTheFigure\thefigure 
 \let\vec\mathbf 
 %\renewcommand{\thefigure}{\theproblem.\arabic{figure}} 
 %\renewcommand{\thefigure}{\theproblem} 
 %\setlist[enumerate,1]{before=\renewcommand\theequation{\theenumi.\arabic{equation}} 
 %\counterwithin{equation}{enumi} 
  
  
 %\renewcommand{\theequation}{\arabic{subsection}.\arabic{equation}} 
  
 %\def\putbox#1#2#3{\makebox[0in][l]{\makebox[#1][l]{}\raisebox{\baselineskip}[0in][0in]{\raisebox{#2}[0in][0in]{#3}}}} 
 %     \def\rightbox#1{\makebox[0in][r]{#1}} 
 %     \def\centbox#1{\makebox[0in]{#1}} 
 %     \def\topbox#1{\raisebox{-\baselineskip}[0in][0in]{#1}} 
 %     \def\midbox#1{\raisebox{-0.5\baselineskip}[0in][0in]{#1}} 
  
 \vspace{3cm} 
  
 \title{ Software Assignment  
  
         \Large AI1110: Probability and Random Variables 
 } 
 \author{ Dhana Lakshmi
          
         AI22BTECH11012 
 } 
  
 \maketitle 
  
 \bigskip 
 \renewcommand{\thefigure}{\theenumi} 
 \renewcommand{\thetable}{\theenumi} 
 
 \begin{enumerate}[label=(\alph*)] 
 \item Introduction \\
 The project, plays audio files from the playlist randomly using a python code. In this we used the moviepy.editor library to convert video files to audio files, pygame library for audio playback, pygame.mixer.music.load() function is used to load the current sound file, pygame.mixer.music.play() function is used to play the loaded sound file, os for importing the files, random allows generation of random numbers and shuffling of songs.
 \item Implementation \\
 This entire project uses python and the steps followed are,\\
 $1)$ Extraction of audio: python code for converting video files to audio files \\ 
 The link for this is: \\
    \url{https://github.com/KotaDhanaLakshmi/probability-and-random-variables/blob/main/software%20assignment/songs.py} \\
  $2)$ File selection and Randomization: After providing path of folder containing files, the program lists of all the audio files and shuffles them randomly \\
  The link for this is: \\
 \url{https://github.com/KotaDhanaLakshmi/Probability-and-Random-variables/blob/main/Software%20Assignment/playlist.py}\\
   pygame.mixer: The library used to load and play the audio files.
\item Usage \\
To use this program these steps are to be followed: \\
$1)$ Run the python program to extract the audio. \\
$2)$ Provide the path to the folder containing audio files and Randomize these audio files. \\
$3)$ Play the songs after shuffling using python code. \\
\item Conclusion \\
Overall, the code plays the sound files in a random order from the specified folder using the Pygame module. It ensures that each sound file finishes playing before moving on to the next one.And plays all songs from playlist without repition.\\
	    
 \end{enumerate} 
 \end{document}
